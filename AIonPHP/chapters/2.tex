\section{Sejarah Machine Learning}
Sejak pertama kali komputer diciptakan manusia sudah memikirkan bagaimana caranya agar komputer dapat belajar dari pengalaman. Hal tersebut terbukti pada tahun 1952, Arthur Samuel menciptakan 
sebuah program, game of checkers, pada sebuah komputer IBM. Program tersebut dapat mempelajari gerakan untuk memenangkan permainan checkers dan menyimpan gerakan tersebut kedalam memorinya.
Istilah machine learning pada dasarnya adalah proses komputer untuk belajar dari data (learn from data). Tanpa adanya data, komputer tidak akan bisa belajar apa-apa. Oleh karena itu jika kita ingin belajar machine learning, pasti akan terus berinteraksi dengan data. Semua pengetahuan machine learning pasti akan melibatkan data. Data bisa saja sama, akan tetapi algoritma dan pendekatan nya berbeda-beda untuk mendapatkan hasil yang optimal.
\begin{enumerate}
	\item pembelajaran terarah (Supervised Learning)
	\item pembelajaran tak terarah (Unsupervised Learning)
	\item Pembelajaran semi terarah (Semi-supervised Learning)
	\item Reinforcement Learning
\end{enumerate}
\section{Dampak Machine Learning di Masyarakat}
Penerapan teknologi machine learning mau tidak mau pasti telah dirasakan sekarang. Setidaknya ada dua dampak yang saling bertolak belakang dari pengembangan teknolgi machine learning. Ya, dampak positif dan dampak negatif.Salah satu dampak positif dari machine learning adalah menjadi peluang bagi para wirausahawan dan praktisi teknologi untuk terus-menerus berkarya dalam mengembangkan sebuah bidang teknologi machine learning. Terbantunya aktivitas yang harus dilakukan manusia pun menjadi salah satu dampak positif machine learning. Sebagai contohnya adalah adanya fitur pengecekan ejaan untuk tiap bahasa pada Microsoft Word. Pengecekan secara manual akan memakan waktu berhari-hari dan melibatkan banyak tenaga untuk mendapatkan penulisan yang sempurna. Tapi dengan bantuan fitur pengecekan ejaan tersebut, secara real-time kita bisa melihat kesalahan yang terjadi pada saat pengetikan.
Akan tetapi disamping itu ada dampak negatif yang harus kita waspadai. Adanya pemotongan tenaga kerja karena pekerjaan telah digantikan oleh alat teknologi machine learning adalah suatu permasalahan yang harus dihadapi. Ditambah dengan ketergantungan terhadap teknologi akan semakin terasa. Manusia akan lebih terlena oleh kemampuan gadget-nya sehingga lupa belajar untuk melakukan suatu aktivitas tanpa bantuan teknologi.
